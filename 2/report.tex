\documentclass{paper}

%\usepackage{times}
\usepackage{epsfig}
\usepackage{graphicx}
\usepackage{amsmath}
\usepackage{amssymb}
\usepackage{color}
\usepackage{subcaption}
\usepackage{caption}


% load package with ``framed'' and ``numbered'' option.
%\usepackage[framed,numbered,autolinebreaks,useliterate]{mcode}

% something NOT relevant to the usage of the package.
\setlength{\parindent}{0pt}
\setlength{\parskip}{18pt}






\usepackage[latin1]{inputenc} 
\usepackage[T1]{fontenc} 

\usepackage{listings} 
\lstset{% 
   language=Matlab, 
   basicstyle=\small\ttfamily, 
} 



\title{Report Computer Vision Project 2}



\author{Single Michael\\08-917-445}
% //////////////////////////////////////////////////


\begin{document}



\maketitle


\section{Binocular Stereo (Due on 28/10/2014)}



\subsection{Epipolar lines Estimation (50 points)}
For given two images - called left and right - both showing a certain object from different viewing positions (i.e. Multiview geometry) we were supposed to estimate the epipolar lines in both images. Epipolar lines are the intersections of epipolar plane with the image plane. Furthermore, when selecting a point from the left image, our program should be able to estimate the corresponding point in the right image. \\

Initially, a user has to select a certain point in the left image and then mark the corresponding position in the right image by clicking at the appropriate position. For each click a user performs we store the corresponding image coordinates at the position where the user performed its click action and also remember from which image this click came from. We store these image coordinates as two dimensional homogenous positions. Remember that two-dimensional homogenous coordinates that represents a point in space have three components, where the last is equal one and the first two correspond to the regular cartesian coordinates of the clicked image position. A user has to mark in both images eight positions using our GUI, alternating, first in the left, then in the right image the corresponding position. The collected homogenous coordinates resulting from the selection-process in the left image are both stored in a $3 \times 8$ matrix called \emph{left} (each column represents a homogenous point). Similarly, the collected selections in the right image are stored in a $3 \times 8$ matrix called \emph{right}. \\

Using these \emph{left} and \emph{right} positions we compute (estimate) the fundamental matrix $F$ by applying the \emph{normalized eight point algorithm} on them. This algorithm is called \emph{normalized} since all selected positions from the left image and those from the right image are centred towards their mean center which is supposed to denote the approximate image origin $(0,0)$ (each image has its own origin). \\ 

The normalization process for a given set of points works as the following: using the (x,y) from a set of points (in our case a matrix carrying homogenous coordinates), we compute the mean position by summing all positions and dividing by the number of used points (in this exercise this is equal to eight points). Then, we shift all the given point positions towards this center by subtracting each position by their mean position. \\

In order to increase the numerical stability of the eight-point algorithm, we are supposed to ensure, that the mean squared distance between the origin (i.e. the mean position of the point set) and each point is equal to 2 pixels. \\

Thus, we also have to rescale each point in the left and in the right position collection by a factor $s = \frac{\sqrt{2}}{\bar{d}}$ where $\bar{d}$ denotes the average distance of all point distances (of the points in a certain point set) to their average position (i.e. their origin). Since we are using homogenous coordinates, we can formulate this \emph{position normalization} as a homogenous transformation, capturing a translation towards the average position, and a rescaling such that the mean squared distance to the origin is equal to two pixels. Formally, such a transformation looks like the following:

\begin{align*}
T = \left(\begin{array}{ccc}
s & 0 & -c_1 s \\
0 & s & -c_2 s \\
0 & 0 & 1 \\
\end{array} \right)
\end{align*}


Where the scaling factor $s$ is the distance scaling (such that the mean squared distance of each position to the origin is equal two pixels) and $(-s c_1, -s c_2)$ is the translation to the origin. \\

For both point sets, \emph{left} and \emph{right}, we have to compute such a homogenous transformation matrix. Then we have to apply these transformations to their points. We denote the normalization transformation for the left point set as $T_{l}$ and for the right point set as $T_{r}$. \\

Next, let us consider the two points $\textbf{x} = (u,v,1)^{T}$ and $\textbf{x'} = (u',v',1)^{T}$ where $\textbf{x}$ is a normalized point from the left image (using $T_l$) and $\textbf{x'}$ a normalized point from the right image. \\

Given the following epipolar constraint 

\begin{align}
    \textbf{x'}^{T} F \textbf{x} = 0
\label{eq:epipolar_constraint}
\end{align}

Where $F$ denotes the (yet) unknown Fundamental matrix we are looking for. By solving the minmizing problem 
\begin{align*}
    \min_{\bf{F}} \sum^{N}_{k=1} \left(\textbf{x'}_{k}^{T} \textbf{F} \textbf{x}_{k}\right)^2 s.t. ||\textbf{F}||^2 = 1
\end{align*}
 We we can therefore retrieve $F$. \\
 In the following I describe some further steps in detail how to compute the fundamental matrix $F$ using above's insights. \\

We start by explicitly showing all components in the epipolar constraint from equation $\ref{eq:epipolar_constraint}$: 

\begin{align*}
    s = s
\end{align*}

 




\subsection{Model reconstruction (50 points)}


\end{document}